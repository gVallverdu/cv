%%%%%%%%%%%%%%%%%
% This is an sample CV template created using altacv.cls
% (v1.1.5, 1 December 2018) written by LianTze Lim (liantze@gmail.com). Now compiles with pdfLaTeX, XeLaTeX and LuaLaTeX.
%
%% It may be distributed and/or modified under the
%% conditions of the LaTeX Project Public License, either version 1.3
%% of this license or (at your option) any later version.
%% The latest version of this license is in
%%    http://www.latex-project.org/lppl.txt
%% and version 1.3 or later is part of all distributions of LaTeX
%% version 2003/12/01 or later.
%%%%%%%%%%%%%%%%

%% If you need to pass whatever options to xcolor
\PassOptionsToPackage{dvipsnames}{xcolor}

%% If you are using \orcid or academicons
%% icons, make sure you have the academicons
%% option here, and compile with XeLaTeX
%% or LuaLaTeX.
% \documentclass[10pt,a4paper,academicons]{altacv}

%% Use the "normalphoto" option if you want a normal photo instead of cropped to a circle
% \documentclass[10pt,a4paper,normalphoto]{altacv}

% \documentclass[10pt,a4paper,normalphoto,academicons]{altacv}
\documentclass[11pt,a4paper,ragged2e]{../alta_letter}

%% AltaCV uses the fontawesome and academicon fonts
%% and packages.
%% See texdoc.net/pkg/fontawecome and http://texdoc.net/pkg/academicons for full list of symbols. You MUST compile with XeLaTeX or LuaLaTeX if you want to use academicons.

% Change the page layout if you need to
\geometry{%
left=2cm,%
right=2cm,%
top=2.5cm,%
bottom=2.5cm}

% Change the font if you want to, depending on whether
% you're using pdflatex or xelatex/lualatex
\ifxetexorluatex
  % If using xelatex or lualatex:
  \setmainfont{Ubuntu}
  \usepackage{polyglossia}
  % \setmainfont{Lato}
  \setdefaultlanguage{french}
  \setmainlanguage{french}
\else
  % If using pdflatex:
  \usepackage[utf8]{inputenc}
  \usepackage[T1]{fontenc}
  \usepackage[default]{lato}
\fi




% Change the colours if you want to
\definecolor{Mulberry}{HTML}{72243D}
\definecolor{SlateGrey}{HTML}{2E2E2E}
\definecolor{LightGrey}{HTML}{666666}
\definecolor{GoodSamaritan}{HTML}{3C6382}
\colorlet{heading}{GoodSamaritan}
% \colorlet{heading}{Sepia}
% \colorlet{accent}{Mulberry}
\colorlet{accent}{GoodSamaritan}
\colorlet{emphasis}{SlateGrey}
\colorlet{body}{SlateGrey}

% Change the bullets for itemize and rating marker
% for \cvskill if you want to
\renewcommand{\itemmarker}{{\small\textbullet}}
\renewcommand{\ratingmarker}{\faCircle}

%% sample.bib contains your publications
% \addbibresource{sample.bib}

\begin{document}
\name{}
% \tagline{Researcher in molecular simulations}
\tagline{}
% \photo{3.4cm}{IPREM-gvallver_carre.jpg}
% \photo{2.8cm}{../img/gvallver-red.jpg}
\personalinfo{%
  % Not all of these are required!
  % You can add your own with \printinfo{symbol}{detail}
  % 1983 August 10 -- Married, 2 children\\
\vspace{1ex}
\mailaddress{62 rue Bourgneuf, 64610 Morlaàs}
\email{germain.vallverdu@gmail.com}
\phone{06 88 59 08 87}
% \mailaddress{IPREM Technopôle Hélioparc, 2 ave du Président P. Angot, FR-64053 Pau cedex 9}
  % \location{Pau, FRANCE}
  % \homepage{gsalvatovallverdu.gitlab.io}
  % \twitter{@twitterhandle}
  % \github{github.com/gvallverdu}
  %% You MUST add the academicons option to \documentclass, then compile with LuaLaTeX or XeLaTeX, if you want to use \orcid or other academicons commands.
  % \orcid{orcid.org/0000-0003-1116-8776}
  % \linkedin{linkedin.com/in/germain-salvato-vallverdu-398b31b2}
}


{\color{emphasis}\Large\bfseries Germain \textsc{Salvato Vallverdu}}
\hfill\today\par

\parbox{.5\textwidth}{\makecvheader}

\justify

\cvsection{~}

\setlength{\parindent}{0pt}
\setlength{\parskip}{1.5ex plus 0.5ex minus 0.5ex}

\bigskip

Madame, Monsieur

\medskip

Je vous adresse ma candidature en réponse à l'offre de poste HPC Software Developper Engineer, référence 16602BR.

Actuellement maître de conférences à l'université de Pau et des Pays de l'Adour, j'ai dix ans d'expérience en simulations numériques et calcul scientifique dans le domaine de la chimie-physique. Curieux de nature, au travers de mes activités de recherche j'ai eu l'opportunité de contribuer à des projets appartenant à des thématiques variées : des systèmes biologiques et l'étude de propriétés photophysiques dans des protéines fluorescentes ; des matériaux d'électrodes pour batteries Li-ion et la réactivité chimique aux interfaces ; les bruts pétroliers et la caractérisation chimique des processus d'agrégations.

À travers l'application de ces projets, j'ai su m'approprier les particularités du domaine pour mettre en place un dialogue constructif au sein d'équipes pluridisciplinaires et encadrer les doctorants ou post-doctorants du projet. J'ai mis mon exigence au service des problématiques rencontrées pour concevoir des modèles pertinents, mettre en œuvre les simulations numériques associées et développer les programmes de calculs nécessaires au traitement ou à la réalisation de ces simulations.

Je souhaite aujourd'hui mettre mon expérience et mes compétences au profit de nouveaux projets au sein du groupe Total. Ce choix est une évidence pour deux raisons. Premièrement, Total est un leader mondial du secteur de l'énergie. Or, la production et la distribution d'énergie représentent un enjeu majeur du XXIe siècle tant sur le plan sociétal qu'environnemental. Deuxièmement, Total est depuis de nombreuses années un acteur international majeur du calcul scientifique haute performance aussi bien sur le plan matériel, avec la machine Pangea classée parmi les supercalculateurs les plus puissants du monde, que sur le plan du développement de logiciels. Ces deux aspects constituent un contexte particulièrement stimulant dans lequel je souhaite relever de nouveaux défis et apporter un regard original de par mon cursus.

Dans l'attente de vous rencontrer, veuillez agréer, Madame, Monsieur, l'expression de mes sincères salutations.

\bigskip

\hspace{.6\textwidth}Germain Salvato Vallverdu

\hspace{.6\textwidth}\includegraphics[width=4cm]{../maSignature}


\end{document}
