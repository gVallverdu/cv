%%%%%%%%%%%%%%%%%
% This is an sample CV template created using altacv.cls
% (v1.1.5, 1 December 2018) written by LianTze Lim (liantze@gmail.com). Now compiles with pdfLaTeX, XeLaTeX and LuaLaTeX.
%
%% It may be distributed and/or modified under the
%% conditions of the LaTeX Project Public License, either version 1.3
%% of this license or (at your option) any later version.
%% The latest version of this license is in
%%    http://www.latex-project.org/lppl.txt
%% and version 1.3 or later is part of all distributions of LaTeX
%% version 2003/12/01 or later.
%%%%%%%%%%%%%%%%

%% If you need to pass whatever options to xcolor
\PassOptionsToPackage{dvipsnames}{xcolor}

%% If you are using \orcid or academicons
%% icons, make sure you have the academicons
%% option here, and compile with XeLaTeX
%% or LuaLaTeX.
% \documentclass[10pt,a4paper,academicons]{altacv}

%% Use the "normalphoto" option if you want a normal photo instead of cropped to a circle
% \documentclass[10pt,a4paper,normalphoto]{altacv}

% \documentclass[10pt,a4paper,normalphoto,academicons]{altacv}
\documentclass[11pt,a4paper,ragged2e]{../alta_letter}

%% AltaCV uses the fontawesome and academicon fonts
%% and packages.
%% See texdoc.net/pkg/fontawecome and http://texdoc.net/pkg/academicons for full list of symbols. You MUST compile with XeLaTeX or LuaLaTeX if you want to use academicons.

% Change the page layout if you need to
\geometry{%
left=2cm,%
right=2cm,%
top=2.5cm,%
bottom=2.5cm}

% Change the font if you want to, depending on whether
% you're using pdflatex or xelatex/lualatex
\ifxetexorluatex
  % If using xelatex or lualatex:
  \setmainfont{Ubuntu}
  \usepackage{polyglossia}
  % \setmainfont{Lato}
    \setdefaultlanguage{french}
    \setmainlanguage{french}

\else
  % If using pdflatex:
  \usepackage[utf8]{inputenc}
  \usepackage[T1]{fontenc}
  \usepackage[default]{lato}
\fi



\usepackage{setspace}

% Change the colours if you want to
\definecolor{Mulberry}{HTML}{72243D}
\definecolor{SlateGrey}{HTML}{2E2E2E}
\definecolor{LightGrey}{HTML}{666666}
\definecolor{GoodSamaritan}{HTML}{3C6382}
\colorlet{heading}{GoodSamaritan}
% \colorlet{heading}{Sepia}
% \colorlet{accent}{Mulberry}
\colorlet{accent}{GoodSamaritan}
\colorlet{emphasis}{SlateGrey}
\colorlet{body}{SlateGrey}

% Change the bullets for itemize and rating marker
% for \cvskill if you want to
\renewcommand{\itemmarker}{{\small\textbullet}}
\renewcommand{\ratingmarker}{\faCircle}

%% sample.bib contains your publications
% \addbibresource{sample.bib}

\begin{document}
\name{}
% \tagline{Researcher in molecular simulations}
\tagline{}
% \photo{3.4cm}{IPREM-gvallver_carre.jpg}
% \photo{2.8cm}{../img/gvallver-red.jpg}
\personalinfo{%
  % Not all of these are required!
  % You can add your own with \printinfo{symbol}{detail}
  % 1983 August 10 -- Married, 2 children\\
\vspace{1ex}
\mailaddress{62 rue Bourgneuf, 64610 Morlaàs, France}
\email{germain.vallverdu@gmail.com}\\
\phone{+33 6 88 59 08 87}
% \mailaddress{IPREM Technopôle Hélioparc, 2 ave du Président P. Angot, FR-64053 Pau cedex 9}
  % \location{Pau, FRANCE}
  % \homepage{gsalvatovallverdu.gitlab.io}
  % \twitter{@twitterhandle}
  % \github{github.com/gvallverdu}
  %% You MUST add the academicons option to \documentclass, then compile with LuaLaTeX or XeLaTeX, if you want to use \orcid or other academicons commands.
  % \orcid{orcid.org/0000-0003-1116-8776}
  % \linkedin{linkedin.com/in/germain-salvato-vallverdu-398b31b2}
}


{\color{emphasis}\Large\bfseries Germain \textsc{Salvato Vallverdu}}
\hfill\today\par

\parbox{.5\textwidth}{\makecvheader}

\justify

\vspace{-5mm}
\cvsection{~}

\setlength{\parindent}{0pt}
\setlength{\parskip}{1.5ex plus 0.5ex minus 0.5ex}

\bigskip

Dear Madam, Dear Sir,

\medskip
\onehalfspacing

I am sending you my application in response to the job offer for Developer AI, reference 19640BR.

Currently as an associate professor at the University of Pau and Pays de l'Adour, I have ten years' experience in numerical simulations and scientific programming in the field of computational chemical-physics as well as high-performance computing. Inquisitive, through my research activities, I contribute to projects linked to miscellaneous scientific domains, such as biological systems and the study of photophysical properties in fluorescent proteins; electrode materials for Li-ion batteries and the chemical reactivity at interfaces; crude oil and the chemical characterization of aggregation processes.

Through the implementation of these projects, I acquired the specificities of the fields to set up and to conduct fruitful exchanges within multidisciplinary teams and supervise the Phd students or post-doctoral researchers of the project. Adapting my skills to each encountered problematic, I designed relevant models, implemented the associated numerical simulations and in particular developed the scientific programs necessary for the production including the analyses and the interpretation of the results obtained from these simulations.

 
Today, I would like to use my experience and skills for the benefit of new projects within the Total Group. This choice is obvious for two reasons. Firstly, Total is a world leader in the energy sector. Energy production and distribution is a major challenge of the 21st century in both societal and environmental terms. Secondly, Total has for many years been a major international actor in high-performance scientific computing both in terms of hardware, with the Pangea machine ranked among the most powerful supercomputers in the world, and in terms of the development. These two aspects constitute a particularly stimulating context in which I wish to take up new challenges and bring an original perspective from my background.

Yours faithfully.

\vspace{1cm}

\hspace{.6\textwidth}Germain Salvato Vallverdu

\hspace{.6\textwidth}\includegraphics[width=4cm]{../maSignature}


\end{document}
