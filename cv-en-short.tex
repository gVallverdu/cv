% -- Encoding UTF-8 without BOM
% -- XeLaTeX => PDF (BIBER)

\documentclass{cv-style}     % Add 'print' as an option into the square bracket to remove colours from this template for printing.

\setdefaultlanguage[variant=british]{english}
\sethyphenation[variant=british]{english}{} % Add words between the {} to avoid them to be cut

%----------------------------------------------------------------------------------------
%	Page layout
%----------------------------------------------------------------------------------------
\cvheadheight{3.5cm}
\cvasidewidth{4.7}
\cvasidevpos{3.5}
\cvmainwidth{11.5cm}
\geometry{left=6.4cm, top=2.5cm, right=1cm, bottom=5mm}

%----------------------------------------------------------------------------------------
%	Bibliography
%----------------------------------------------------------------------------------------
\usepackage[sectcntreset]{bibtopic}
\usepackage{natbib}
\bibliographystyle{bib/achemso_perso}
\AtBeginDocument{\nocite{achemso-control}}
%\setlength{\bibsep}{2pt plus 0.3ex}

%----------------------------------------------------------------------------------------
%	hyperlink setup
%----------------------------------------------------------------------------------------
\hypersetup{
    pdftitle=Resume \textbar{} Germain Vallverdu,%
    pdfauthor=Germain Vallverdu
}

%----------------------------------------------------------------------------------------
%	Setup las updated text
%----------------------------------------------------------------------------------------
%\lastupdated{Last Updated on \today}

%----------------------------------------------------------------------------------------
%	Add a few custom packages
%----------------------------------------------------------------------------------------
\usepackage{fontawesome}

% \usepackage{academicons}
% \definecolor{orcidlogocol}{HTML}{A6CE39}

\begin{document}

\header{Germain }{Vallverdu}{Associate Professor - Chemical physics and numerical simulations}          % Your name

%----------------------------------------------------------------------------------------
%	SIDEBAR SECTION  -- In the aside, each new line forces a line break
%----------------------------------------------------------------------------------------

\begin{aside}
    \includegraphics[width=.8\columnwidth]{img/gvallver}
    10 août 1983, France
    Maried, 2 children
    %
    \section{Contact}
    germain.vallverdu@univ-pau.fr
    (33) 5 59 40 78 51
    (33) 6 88 59 08 87
    ~
    IPREM
    Technopôle Hélioparc
    2 ave du Président P. Angot
    FR-64053 Pau cedex 9
    \href{http://iprem.univ-pau.fr/fr/index.html}{\color{blue} \faGlobe{} http://iprem.univ-pau.fr}
    %
    \section{Theoretical Chemistry}
    Computational strategy
    Development
    Surfaces, interfaces
    VASP, CRYSTAL (solid)
    Gaussian, Orca (molecule)
    Gromacs, AMBER (dynamic)
    %
    \section{Programming}
    Fortran, C
    Python
    \LaTeX{}, HTML/CSS
    %
    \section{Languages}
    French
    English (Professional)
    %
    \section{Bibliometry}
    15 articles
    11 conferences
    h-\textit{index}: 7
    12.5 citations per item
    150 citations (142 w/o self-citations)
    %
    \section{On the web}
    \href{http://orcid.org/0000-0003-1116-8776}{$\vcenter{\hbox{\includegraphics[height=4mm]{img/orcid_128x128}}}$\small ~orcid.org/0000-0003-1116-8776}
    %\href{https://github.com/gVallverdu}{$\vcenter{\hbox{\includegraphics[height=16pt]{img/GitHub-Mark}}}$ gVallverdu}
    \href{https://github.com/gVallverdu}{\color{gray}\faGithub{}\small ~gVallverdu}
    \href{http://gvallver.perso.univ-pau.fr}{\color{blue}\faGlobe{}\small ~http://gvallver.perso.univ-pau.fr}
    %
\end{aside}

\vspace{0mm}
\section{Abs}{tract}
\vspace{-0.2cm}


Associate professor at the University of Pau \& Pays Adour, I am a specialist in
theoretical chemistry and numerical simulations at IPREM institute.
My research activities concern the development of new methods and
computational strategies at different time or space scales, applied to the
investigations of complex systems (surfaces, interfaces, complex matrices in
condensed matter). I teach mainly chemical-physics subjects and programming
languages at the university of Pau.

% Associate professor at the Université de Pau et des Pays de l'Adour,
% I am a theoretical chemist at IPREM institute (Institute for Analytical sciences and
% chemical physics applied to environment and materials).
% My research activities concern the development of new methods in theoretical chemistry
% and new computational
% strategies at different time or space scales, applied to the investigations
% of complex systems.
% I teach mainly chemical-physics subjects and programming languages at the university of Pau.

%----------------------------------------------------------------------------------------
%	SKILLS SECTION
%----------------------------------------------------------------------------------------

%\section{skills}
%  \vspace{-0.2cm}

%Skill 1, skill 2, skill 3, skill 4, skill 5.

%----------------------------------------------------------------------------------------
%	WORK EXPERIENCE SECTION
%----------------------------------------------------------------------------------------
\vspace{-2mm}
\section{Professional }{Experiences}
\vspace{-0.3cm}

\begin{entrylist}
%------------------------------------------------
\entry
  {since 2010~}
  {Université de Pau et des Pays de l'Adour}
  {Pau, France}
  {\jobtitle{Associate professor}\\
   Theoretical chemistry and computational approaches.
   Surfaces, interfaces, reactivity and molecular interactions.}
%------------------------------------------------
\entry
  {2009--2010}
  {CEA - DAM}
  {Bruyères le châtel, France}
  {
  \jobtitle{Postdoctoral position}\\
  Development and implementation of a mesoscopic model for reactive shock waves
  propagation in heterogeneous systems.
  }
%------------------------------------------------
\entry
  {2006--2009}
  {Université Paris-Sud 11}
  {Orsay, France}
  {
  \jobtitle{PhD Student}\\
  Theoretical study of photophysical processes in fluorescent proteins.
  }
%------------------------------------------------
\end{entrylist}

%----------------------------------------------------------------------------------------
%	EDUCATION SECTION
%----------------------------------------------------------------------------------------
\vspace{-5mm}
\section{Educ}{ation}
\vspace{-0.2cm}

\begin{entrylist}
%------------------------------------------------
\entry
{2006-2009}
{PhD in chemistry {\normalfont speciality theoretical chemistry}}
{Université Paris-Sud 11}
{Mention très honorable}
%------------------------------------------------
\entry
{2004-2006}
{Master degree of chemistry}
{Université Paris-Sud 11}
{{\normalfont speciality molecular chemical-physics}\par Mention TB}
%------------------------------------------------
\entry
{2003-2004}
{Bachelor Degree of chemical-physics}
{Université Paris-Sud 11}
{Mention TB}
%------------------------------------------------
\entry
{2003-2006}
{Magistère de Physico-Chimie Moléculaire}
{Université Paris-Sud 11 -- ENS Cachan}
{\vspace{-4mm}}
%------------------------------------------------
\entry
{2001-2003}
{Undergraduate {\normalfont physics and chemistry}}
{Lycée François Arago, Perpignan}
{}
\end{entrylist}

%----------------------------------------------------------------------------------------
%	OTHER QUALIFICATIONS SECTION
%----------------------------------------------------------------------------------------
\vspace{-6mm}
\section{Main}{ publications}
\vspace{-0.2cm}
\nocite{aqturin2017_2, santos2017, vallverdu2016, guille2015, Martin2012, Maillet2011, Vallverdu2010}
\begin{btSect}{bib/cv_articles}
    \setlength{\bibsep}{2pt}
    \btPrintCited
\end{btSect}

%----------------------------------------------------------------------------------------
%	INTERESTS SECTION
%----------------------------------------------------------------------------------------

% \section{interests}
%   \vspace{-0.2cm}
%
% \textbf{professional:} professional interest 1, professional interest 2 and professional interest 3.
% \textbf{personal:} personal interest 1, personal interest 2, personal interest 3 and personal interest 4.

%----------------------------------------------------------------------------------------
%	TEACHING SECTION
%----------------------------------------------------------------------------------------
\vspace{-1mm}
\section{Teach}{ing}
  \vspace{-0.3cm}

$\bullet$ Lectures in chemical-physics, theoretical chemistry and programming languages.\\
%in bachelor degrees, master degrees and doctoral school. About 200 h/year\\
$\bullet$ Strong involvement in new information and communication technologies for education\\
$\bullet$ Science popularization: Quantum mechanics and workshops for school students
%organization of animations for secondary and primary school pupils.

\vspace*{-5mm}

%\begin{itemize}
%    \item Lectures in chemical-physics, theoretical chemistry and programming languages, in bachelor
%    degrees, master degrees and doctoral school.
%    \item Science popularization: Quantum mechanics, animations for secondary and primary school pupils
%\end{itemize}

%----------------------------------------------------------------------------------------

\end{document}
